\documentclass{article}

\usepackage{floatrow}
\usepackage{graphicx} % Required for inserting images
\usepackage[a4paper, total={6in, 8in}]{geometry}
\usepackage{parskip} % Required to stop indenting new paragraphs
\usepackage{amssymb}
\usepackage{amsmath}
\usepackage{algpseudocode}
\usepackage{algorithm}
\usepackage[numbers]{natbib} % number references and natbib for more styles
\bibliographystyle{unsrtnat} % sets bibliography to vancouver style

\title{Reverse Mode Algorithmic Differentiation Abstract}
\author{cf1021, sam221, aml21, mf621}
\date{}

\begin{document}

\maketitle

In this project, we have set out to implement Reverse Mode Algorithmic Differentiation, and using our implementation, demonstrate why and how it is used in preference to other methods. We start off by creating Directed Acyclic Graphs (DAGs) from the functions we want to differentiate. We created evaluation functions, traversing the tree, applying the mathematical functions to give us a single output, then traversed backward through the tree and evaluated the adjoint at each elementary operation, in other words, we took the derivative of each elementary operation and multiplied through to, using the chain rule, give us our final derivatives of our original expression.

From there we both proved the accuracy of our code, using a Taylor Test and demonstrated the time efficiency of our algorithm by comparing it to how long the forward mode algorithmic differentiation code to run. We then look at extensions of our implementation and also looking at how we can implement PDEs into our expressions. Notably, the one-dimensional Advection-Diffusion equation. We created time steps to implement an iterative method, using central differences to approximate our derivatives. Then used reverse mode algorithmic differentiation to differentiate the algorithm of our time step.

\end{document}